Machine learning is generally categorized into three categories: supervised, unsupervised and reinforcement learning. The supervised learning focus on reaching a targeted outcome through given features or attributes. Unsupervised learning on the other hand, focus on clustering and grouping. Reinforcement learning adopts various agents with decision matrices and reward function to master a given action. Developing a winning algorithm for board games is a typical reinforcement learning task like the famous alpha Go.
\par

\subsubsection{Supervised Learning}

Recent popularity of deep learning, which largely driven by its superior performance in image pattern recognition applications, is a type of supervised learning models. The supervised learning  derives the relationship between input~\textit{x} and the output~\textit{y}, it comes with two major categories as probabilistic and non-probabilistic supervised learnings. 

\paragraph{Probabilistic based learning}
Most of modern supervised learning strategies, particularly the expansion of deep learning mechanisms, many are considered black-box approach, as we know little about the underlying mechanism how model works. Therefore, it is increasingly important to understand the statistical uncertainty associated with model predictions. Probabilistic based supervised learnings are designed to provide full spectrum of predictive outcomes. 
\par 

In linear regression models, model parameters can be learned using Bayesian estimation. Given model is trained as following

\begin{equation}
    \mathcal{y} = \mathrm{X}\mathcal{\phi}
\end{equation}
where $\mathrm{X}$ is input vector, $\mathcal{y}$ is model output, $\mathcal{\phi}$ is the parameter matrix. If $\mathcal{y}$ expressed as Gaussian conditional distribution, then it gives:

\begin{equation}
    \mathcal{p}(\mathcal{y}|\mathrm{X},\mathcal{\phi}) = \mathcal{N}(\mathcal{y};\mathrm{X}\mathcal{\phi},\mathrm{I})
\end{equation}

the prior distribution of model parameter $\phi$ is commonly approximated by Gaussian, which gives by 

\begin{equation}
    \mathcal{p}(\phi) = \mathcal{N}(\mathcal{\phi}; \mu, \Lambda)
\end{equation}
where $\mu$ and $\Lambda$ corresponds to prior mean and covariance respectively. 

By Bayesian estimation, the posterior distribution of model parameter matrix $\phi$ can be derived as 

\begin{equation}
    \mathcal{p}(\phi|\mathrm{X},\mathcal{y}) \propto \mathcal{p}(\mathcal{y}|\mathrm{X}, \phi)\mathcal{p}(\phi)
\end{equation}

The optimum model parameter matrix $\phi$ can be approximated under Bayesian estimate when prior mean $\mu$ set to zero and covariance $\Lambda$ set to $\frac{1}{\alpha}\mathcal{I}$, while $\alpha$ is the learning rate. In addition to provide optimum model parameter matrix $\phi$, Bayesian based supervised learning algorithm provides a whole spectrum of possible values $\phi$ may adopt, such insight is largely considered an advantage of probabilistic based supervised learning. \citet{Goodfellow-et-al-2016} provided a detailed theoretical study for Bayesian estimation based probabilistic learning algorithm. 
\par 

To generalize into classification models, we can define binary class distribution, namely zero and one. One way to classify a distribution over binary categories is to adopt logistic sigmoid function, which has an output between $(0,1)$ and this can be treated as a probability given as 

\begin{equation}
    \mathcal{p}(y=1|\mathcal{x};\phi) = \psi(\phi^T\mathcal{x})
\end{equation}
where $\phi$ is model parameter matrix and $\psi$ is an sigmoid function. Solving for optimum parameter matrix $\phi$ is somewhat difficult, given that there is no closed-form solution to its optimum. A common approach is to search the parameter space by maximizing the log-likelihood. 

\paragraph{Support Vector Machines (SVMs)}
SVMs are versatile and have wide applications.~\cite{Boser_1992,Cortes:1995:SN:218919.218929} The diversity of SVMs is provided by the versatile form of kernels. Kernels can take forms of linear, non-linear, polynomial, radial basis function, sigmoid and many more. The kernel functions return the inner product, which is computationally favourable in high-dimensional space. Kernels are designed to reduce the feature dimensions and concentrate the weight of representations that close to support vectors. 
\\
SVMs are linear driven models by $\mathcal{w}^T\mathcal{x}+b$. SVMs based models only output class identification as positive class and negative class. Kernel precondition that machine learning algorithms can take form of dot products between examples. Kernel function is then defined as $\mathcal{k}(\mathcal{x},\bar\mathcal{x})$, where $\bar\mathcal{x}$ denotes training dataset. Kernel function can take the form of:

\begin{equation}
    \mathcal{k}(\mathcal{x},\bar\mathcal{x}) = \psi(x) \cdot \psi(\bar\mathcal{x})
\end{equation}
where $\psi(x)$ is a given feature function. 

Kernel tricks enable the study of non-linear relationships with respect to input $\mathrm{X}$ by guarantee a convergence under convex optimization. Furthermore, kernel function $\mathcal{k}$ is computationally more efficient than taking the dot product of two feature functions. 

\paragraph{Other Popular Supervised Learning}
Another popular class of supervised learning algorithms is nearest neighbour regression. Non-parametric $\mathcal{k}$-nearest neighbour is suitable for both regression and classification models and is arguably one of most well-known. $\mathcal{k}$-nearest neighbour algorithm is capable to obtain high accuracy given sufficient large training dataset available in the expense of high computational cost. On the other hand, $\mathcal{k}$-nearest neighbour algorithm is constrained to learn equally pronounced features in underlying data structure, it is unlikely to be successful to learn discriminative and more distant away features, also $\mathcal{k}$-nearest neighbour algorithm is more prone to over fitting issue.


\subsubsection{Unsupervised Learning}
Although the distinction between supervised and unsupervised learnings are not rigid due to the lack of quantitative measurement to quantify, the general consensus for defining unsupervised learnings associates with density estimation, where involve extracting features within dataset~\textit{x} then sorting data points in dataset~\textit{x} into feature categories.
Unsupervised learnings often involve finding a close representation of a big given  dataset, which preserves as much information in dataset~\textit{x} while much slimmer in size. In a broad term, three types of representations are commonly found, namely, sparse representation, independent representation and lower dimensional representation. The sparse representation~\cite{Barlow_1989,Olshausen_1996,Hinton_1997} has higher dimentionality compare to the other two to compensate the sparsity of the data volume, which likely to result the overall representation space concentrates along the representation axes. In independent representation, the representation are selected in such a way that they are statistically independent to each other. For lower dimensional representation, the efforts focus on reducing dimensionality of dataset for but largely preserve features at the same time. 

\paragraph{Principal Components Analysis (PCA)}
PCA is one of the most common data compression algorithm in unsupervised learnings to reduce data dimensionality. This is particular useful when the underlying dataset too high in dimensions to learn efficiently. The result construct can be perceived as a projection of the target dataset $\mathcal{x}$ into $\mathcal{k}$ dimensions and is typically a linear transformation of given dataset $\mathcal{x}$.
\par 
One of the major benefit of PCA is the ability to transform underlying dataset into a representation with uncorrelated components (as shown in Figure~\ref{396105}), which result a lean representation of original dataset. Although correlation is one important relationship between data points within a dataset, to elucidate more complex feature dependencies require additional algorithms. 

\paragraph{$\mathcal{k}$-means Clustering}
In $\mathcal{k}$-means clustering, the training dataset is divided into $\mathcal{k}$ clusters that are close to each other.  $\mathcal{k}$-means clustering often utilise one-hot encoding to represent input dataset $\mathcal{x}$, where one-hot encoding adopts a string of only one high bit (i.e. one) and rest of low bit (i.e.zero) combination in representation~\cite{Harris:2012:DDC:2381028}. Therefore, $\mathcal{k}$-means clustering can be perceived as a representation with $\mathcal{k}$-dimensional one-hot vector $\Gamma$. For example, a data entry $\mathcal{x}_i \in$ cluster $\mathcal{i}$ can be presented by one-hot vector $\Gamma$ with $\gamma_i = 1$, the rest $\mathcal{k}-1$ entries equal to zero.
\par 
One-hot encoding is an extreme example of sparse representation and it offers great computational advantages, althrough due to extreme sparsity, it functions less efficiently in generalisation and support less network deepth~\cite{LeCun_2015} in comparison with distributed representation. In addition, despite the clustering property can be measured in average Euclidean distance within each cluster, clustering methods inherit ill posed problem, where there is hardly any quantitative criteria to measure the performance of the clustering in representing real dataset.  Consequently, distributed representation is currently preferred representation, particularlly in deep neural networks. 
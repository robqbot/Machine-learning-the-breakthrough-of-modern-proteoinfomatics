Machine learning is generally categorized into three categories: supervised, unsupervised and reinforcement learning. The supervised learning focus on reaching a targeted outcome through given features or attributes. Unsupervised learning on the other hand, focus on clustering and grouping. Reinforcement learning adopts various agents with decision matrices and reward function to master a given action. Developing a winning algorithm for board games is a typical reinforcement learning task like the famous alpha Go.
\par

\subsubsection{Supervised Learning}

Recent popularity of deep learning, which largely driven by its superior performance in image pattern recognition applications, is a type of supervised learning models. The supervised learning  derives the relationship between input~\textit{x} and the output~\textit{y}, it comes with two major categories as probabilistic and non-probabilistic supervised learnings. 

\paragraph{Probabilistic based learning}
\paragraph{Support Vector Machines (SVMs)}
SVMs are driven by linear function $\mathcal{w}^T\mathcal{x}+b$ 

\subsubsection{Unsupervised Learning}
Although the distinction between supervised and unsupervised learnings are not rigid due to the lack of quantitative measurement to quantify, the general consensus for defining unsupervised learnings associates with density estimation, where involve extracting features within dataset~\textit{x} then sorting data points in dataset~\textit{x} into feature categories.
Unsupervised learnings often involve finding a close representation of a big given  dataset, which preserves as much information in dataset~\textit{x} while much slimmer in size. In a broad term, three types of representations are commonly found, namely, sparse representation, 
Loss functions are used to measure the discrepancy between model predictions and training datasets. Loss functions broadly divided into two categories: classification and regression. As the cross entropy loss (in machine learning also known as log loss) arguably the most popular loss function choice for classification models that commonly behind the deep learning applications in image recognitions, Mean Square Error (MSE) on the other hand, is probably the most well-known regression loss function. 
\par 
Most of the modern neural networks adopt back propagation algorithm for model training (more about model trainings in later section), the loss function therefore has to be at least 1\textdegree differentiable to allow error gradient to be calculated. The loss function can be generalized in the form:

\begin{equation}
    E = \frac {1}{n} \sum_x^n E_x
\end{equation}

where $E_x$ denotes the error occurred on a training on dataset $x$ with $n$ number of training points. If the error has been measured in the square of Euclidean distance:

\begin{equation}
    E(\mathcal{I}) = \frac {1}{2n} \sum_x^n \parallel y(\mathcal{I}) - y^*(\mathcal{I}) \parallel^2 
\end{equation}

where $\mathcal{x}$ is the input vector, $y^*(\mathcal{I})$ is the actual output with given input vector $\mathcal{I}$, $y(\mathcal{I})$ is the model predicted output with given $\mathcal{I}$. 
\par 
We here discuss a few common loss functions from each classification and regression categories. Although the classification models can be used to predict a continuous value as probability of a class label, classification models more often used for predicting a discrete categories, classes or labels a given input belongs to using past experience. Common loss functions used for classification models include:

\begin{enumerate}
    \item  
    \item Hinge Loss/Multi-class SVM loss
\end {enumerate}

\begin{enumerate}
    \item Mean Squared Error (MSE) \\
MSE has the form:
\begin{equation}
    MSE = \frac{1}{n}\sum_{i=1}^n (\hat{x}_i - x_i)^2
\end{equation}
where $\hat{x}_i$ is the estimator and $x_i$ is the actual value. 
    \item Likehood Loss \\
Likehood loss is commonly used in classification models. Although the the approach is an empirical, it is helpful in comparing model performance. 
    \item Cross Entropy \\
Cross entropy is also known as log loss and is widely applicable in binary classification, which is extremely popular in visual pattern recognition application with convolutional neural network, which has the form:
\begin{equation}
    \mathcal{L} = -\frac{1}{n}\sum_{i=1}^n [y_i \times log\hat{y_i} +(1-y_i) \times log(1-\hat{y_i}) ] 
\end{equation}

The function measure the divergence between the distribution of predicted values and distribution of observed values. One advantage of cross entropy loss is that algorithm provides fast convergence and allows quick learning, this is particularly advantageous for neurons constructed with sigmoid activation functions. 
    \item Other forms\\
There are many loss forms in the literature with their own advantages and disadvantages like Mean Squared Logarithmic Error (MSLE), which adopts the form:

\begin{equation}
    \mathcal{L} = -\frac{1}{n}\sum_{i=1}^n [log (y_i + 1) -  log(\hat{y_i} + 1) ]^2 
\end{equation}

which has advantages in dealing with models that work with large absolute values, such scenario often attracts penalties during model training even the error percentage is low. 
\end{enumerate}

\par 
One advantage of back propagation algorithm is that unlike other method, it can operate on non-normalized input vectors although additional normalization step will generally improve performance. \cite{Buckland:2002} The main theoretical challenge with gradient descent back propagation is that during optimisation, the gradient descent can stuck at local minimum and unable to reach the global minimum, this particularly concerning when error function is non-convex and the error surface is extremely rugged. Nevertheless,~\citet{LeCun_2015} argued this concern is largely theoretical and it is very unlikely to occur in majority practical applications. 
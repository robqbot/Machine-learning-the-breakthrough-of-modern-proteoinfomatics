\section{Introduction}

In the last two decades, high throughput proteomics have dominated the protein research skyline, particularly the mass spectrometry based pipelines. With the development of soft ionisation, delicate material like protein/peptides are now able to sustain the energy and heat resulted from high intensity laser beam and therefore to be analysed precisely through mass spectrometry, which in turn, largely expended our capacity in identifying protein and protein modifications in large scale. At the same time, with the cost of proteomics study decreases, the number of large scale high-throughput proteomics researches skyrocketed, the pace of data generated through wet experiments far exceeded our capacity in analysing them through traditional means, many raw data has left unstudied. Recent developments in machine learning has shown promising performance in wide applications, in this article, we will discuss the current applications of machine learning in proteomics research and debate whether machine learning can be applied for more efficient proteomics data processing and drawing more insights.
\par 


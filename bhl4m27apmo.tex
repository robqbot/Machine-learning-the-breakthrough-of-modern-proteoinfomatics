\section{Introduction}

In the last two decades, high throughput proteomics have dominated the protein research skyline, particularly the mass spectrometry based pipelines. With the development of soft ionisation, delicate material like protein/peptides are now able to sustain the energy and heat resulted from high intensity laser beam and therefore to be analysed precisely through mass spectrometry, which in turn, largely expended our capacity in identifying protein and protein modifications in large scale. At the same time, with the cost of proteomics study decreases, the number of large scale high-throughput proteomics researches skyrocketed, the pace of data generated through wet experiments far exceeded our capacity in analysing them through traditional means, many raw data has left unstudied. Recent developments in machine learning has shown promising performance in wide applications, particular deep neural networks has drawn large public attntionto efficiently process proteomics data and draw insights 
\par 
Machine learning generally includes three categories as supervised, unsupervised and reinforcement learning. The supervised learning focus on reaching a targeted outcome through given features or attributes. Unsupervised learning on the other hand, focus on clustering and grouping. Reinforcement learning adopts various agents with decision matrices and reward function to master a given action; developing a winning algorithm for board games is a typical reinforcement learning like the famous alpha Go.
\par
Recently heated deep learning, largely due to its superior performance in image pattern recognition, is a type of artificial neural network, which belongs to the realm of supervised learning. 



\paragra{$L^1$ and $L^2$ Regularization}

$L^1$ and $L^2$ are among the preferred choices, but there is the noticeable difference between the two. $L_1$ commonly known as LASSO regularization (least absolute shrinkage and selection operator), which was firstly introduced by~\citet{Santosa_1986}, later popularized by~\citet{Tibshirani_1996} and adopts form:

\begin{equation}
   \Omega(\phi) = \frac{\lambda}{m} \sum_i^m | \mathcal{w_i} |
\end{equation}

where $\mathcal{w}$ indicates all node weights that are affected by the regularization. The modulus | \mathcal{w} | is the penalty term. $\lambda$ is a pre-selected free parameter that determines the degree of penalty for over complex neuron networks. 
\par 
The gradient of $L^1$ regularization is non-linear sign function $sign (\mathcal{w})$, which does not has a trivial algebraic solution for the optimisation function $\mathcal{Obj}(\mathcal{X},\mathcal{y},\phi)$. Practically, one main difference between $L^1$ and $L^2$ is that $L^1$ can penalise the weight of a neuron to zero whereas $L^2$ penalise the weight towards zero but not equal to zero. In another words, $L^1$ tend to result a more sparse model and for this reason, $L^2$ is generally prefer over $L^1$ unless we purposefully compress model.

$L^2$ has the form:
\begin{equation}
   \Omega(\phi) = \frac{\lambda}{2m} \sum_i^m | \mathcal{w_i} |^2
\end{equation}
$L^2$ regularization also commonly referred as weight decay, ridge regression or Tikhonov regularization. The optimum $\mathcal{w}$ in $L^2$ regularization can be solved algebraically with the help of positive semi-definite $\mathcal{H}$~\cite{Goodfellow-et-al-2016}, so called Hessian matrix. The detailed derivation has been detailed by~\citet{Goodfellow-et-al-2016}. The optimum $L^1$ and $L^2$ regularization can also be viewed as boundary conditions for $arg (min (\Omega)) $ at $\sum \parallel \mathcal{w} \parallel < t $ and  $\sum \parallel \mathcal{w} \parallel^2 < t $ respectively, where $t$ is the predetermined threshold level. Therefore the boundary will be plotted like Figure~\ref{542725} and the optimum solution of $Obj$ thus will be obtained at the touch of boundary contour of $L^1$ or $L^2$. 
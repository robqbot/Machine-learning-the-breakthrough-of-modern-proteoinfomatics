\paragraph {Loss Functions for Regression models}
Regressional models predict continuous values
\begin{enumerate}
    \item Mean Squared Error (MSE) \\
MSE commonly referred as L2 loss and arguably the most popular form of loss function in regression models, which adopts the form:
\begin{equation}
    MSE = \frac{1}{n}\sum_{i=1}^n (\hat{x}_i - x_i)^2
\end{equation}
where $\hat{x}_i$ is the estimator and $x_i$ is the actual value. MSE is continuous and differentiable, which makes MSE particularly desirable for computation as gradient can be easily calculated. 
    \item Mean Absolute Error (MAE) \\
MAE is referred as L1 loss in statistics, similar to MSE, MAE is also a common form to measure model fitness with slight mathematical variation:
\begin{equation}
    \mathcal{L} = \frac {1}{n} \sum_{i=1}^n |y_i - \hat{y}_i|
\end{equation}
This variation however made MAE is difficult to calculate gradient and heavy on computation. Nevertheless, due to its linear representation, MAE does offer balanced penalty on outlier errors, MSE on the other hand, can be easily dominated by the heavy penalty on this region, which may create undesired penalty profile. As a thumb of rule, MAE is more tolerant to outlier prediction and therefore MAE can be used in scenario where outlier is present in the population. MAE particularly useful if training data is contaminated and outlier is abundant. 
    \item Huber Loss\\
Huber loss is introduced by~\citet{Huber_1964}, which is also known as smooth absolute error. Huber loss hybrids the advantages on both differentiability and tolerance on outlier, which is particularly useful in examine datasets with distinct cluster substructures. For example, a collection of local temperature profile over winter and summer with winter data consists majority; under which case, MAE lose enabled model will likely to tuning exclusively to winter profile due to its insensitivity to outlier, but MSE loss enabled models will likely to skew towards summer profile. Both MAE and MSE loss function enabled models are unable to provide clear insights about clustered structure. 
\par 
Huber loss adopts piecewise approach as: 

\begin{equation}
    \mathcal{L}_\delta = \left\{ 
    \begin{array}{ll}
        \frac {1}{2} (y - \hat{y})^2 & \text{for } |y-\hat{y}| \leq \delta, \\
        \delta |y - \hat{y}| - \frac{1}{2} \delta^2 & \text{otherwise.} 
    \end{array}\right.
\end{equation}
The advantage of Huber loss is apparent at small error region, where Huber loss particular the
 
    \item Other forms\\
There are many loss forms in the literature with their own advantages and disadvantages like Mean Squared Logarithmic Error (MSLE), which adopts the form:

\begin{equation}
    \mathcal{L} = -\frac{1}{n}\sum_{i=1}^n [log (y_i + 1) -  log(\hat{y_i} + 1) ]^2 
\end{equation}

which has advantages in dealing with models that work with large absolute values, such scenario often attracts penalties during model training even the error percentage is low. 
\end{enumerate}

\par 
One advantage of back propagation algorithm is that unlike other method, it can operate on non-normalized input vectors although additional normalization step will generally improve performance. \cite{Buckland:2002} The main theoretical challenge with gradient descent back propagation is that during optimisation, the gradient descent can stuck at local minimum and unable to reach the global minimum, this particularly concerning when error function is non-convex and the error surface is extremely rugged. Nevertheless,~\citet{LeCun_2015} argued this concern is largely theoretical and it is very unlikely to occur in majority practical applications. 
\paragraph {Loss Functions for Regression models}
Regressional models predict continuous values
\begin{enumerate}
    \item Mean Squared Error (MSE) \\
MSE has the form:
\begin{equation}
    MSE = \frac{1}{n}\sum_{i=1}^n (\hat{x}_i - x_i)^2
\end{equation}
where $\hat{x}_i$ is the estimator and $x_i$ is the actual value. 
    \item Mean Absolute Error (MAE) \\
Similar to MSE, MAE is also a common form of measure model fitness with slight mathematical variation:
\begin{equation}
    \mathcal{L} = \frac {1}{n} \sum_{i=1}^n |y_i - \hat{y}_i|
\end{equation}
This variation however made MAE is difficult to calculate gradient and heavy on computation. Nevertheless, due to its linear representation, MAE does offer balanced penalty on outlier errors, MSE on the other hand, can be easily dominated by the heavy penalty on this region, which may create unde
    \item \\
    \item Other forms\\
There are many loss forms in the literature with their own advantages and disadvantages like Mean Squared Logarithmic Error (MSLE), which adopts the form:

\begin{equation}
    \mathcal{L} = -\frac{1}{n}\sum_{i=1}^n [log (y_i + 1) -  log(\hat{y_i} + 1) ]^2 
\end{equation}

which has advantages in dealing with models that work with large absolute values, such scenario often attracts penalties during model training even the error percentage is low. 
\end{enumerate}

\par 
One advantage of back propagation algorithm is that unlike other method, it can operate on non-normalized input vectors although additional normalization step will generally improve performance. \cite{Buckland:2002} The main theoretical challenge with gradient descent back propagation is that during optimisation, the gradient descent can stuck at local minimum and unable to reach the global minimum, this particularly concerning when error function is non-convex and the error surface is extremely rugged. Nevertheless,~\citet{LeCun_2015} argued this concern is largely theoretical and it is very unlikely to occur in majority practical applications. 
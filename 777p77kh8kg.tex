Deep neural network has shown promising ability in pattern recognitions, particular visual interpretation.
\par 
The importance of the protein structure lies not only to the fundamental understanding of biological and pharmacological functions, but also have profound clinical diagnostic applications. Protein misfolding contributes to many high profile diseases including circulatory malfunctions like sickle cell disease; chronic neural malfunctions like Alzheimer, Parkinson and Huntinton's disease; chronic autoimmune complications like cystic fibrosis and many more.~\cite{Hammarstrom_2003,Chiti_2006,Selkoe_2003}  
\par 

\subsection{Protein Structure Prediction}
Predicting protein folding is challenging, largely due to it large degrees of freedom, the possibilities are astronomical for even a small protein. As Levinthal's paradox described~\cite{LEV69}, the protein folding possibility is so great that if a protein with 100 amino acid residues to explore all possible conformation sequentially but quickly before arrive to its correct 3D construction, time spent would surpass the age of the universe. Although early attempt by~\citet{Zwanzig_1992} that adopts physical energy bias modelling has bought the Levinthal's paradox to a conclusion, the accuracy and robustness of early models were not in par with scientific requirements.  
\par 
With the rise of deep learning, there are many protein folding predictions based on various deep neural networks have emerged. \citet{Lyons_2014} proposed the first auto-encoder deep neural network to predict C$\alpha$ backbone angles and dihedral bond angles.~\citet{Heffernan_2015} has further proposed with modern deep neural networks yield some great results. The 

One of the recent advancement in predicting protein 3D structures is delivered by AlphaFold
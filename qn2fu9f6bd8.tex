\subsection{Concepts of ANN models}
One of the learning algorithm is artificial neural networks (ANN) as demonstrated in Figure~\ref{147379}. The very first of ANN was proposed by~\citet{McCulloch_1943}, which was inspired by the structure of neural connections. Since 1960s, the discovery about visual information is processed through a layer by layer fashion in cats visual cortex~\cite{Hubel_1959,Hubel_1962}, several studies~\cite{Weng,Weng_1992,Weng_1997,Wenga,Riesenhuber_1999,Viglione_1970} have developed various ANN models in attempt to extract features from visual objects, however, ANN has only start to show promising performance and potentials since 2006 ~\cite{Hinton_2006a, Hinton_2006_1}. 
\par 
A typical ANN model construction possesses three structural layers, namely: one input layer, one or more hidden layers and one output layer. Each layer can have one or more nodes and each node is commonly referred as a neuron. Neurons in the each layer are either connected or partially connected with neurons at neighbour layers. The output value $O_o$ can be presented in inputs form as 

\begin{equation}
   O_o = g( \sum_{j} \sum_{i} B2*B1*W_{ij} * I_i) 
\end{equation}

where $I_i$ represents the input layer with i nodes; $O_o$ is the output layer with o nodes; $W_{i}$ is the weight of each input node $I_i$ in representing hidden node $H_j$; $W_{j}$ is the weight of each hidden node $H_j$ in representing output node $O_o$;  B1 and B2 are bias layers that apply constant values to the nodes at each subsequent layers after input layer; g is the activation function and generally is a non-linear function, sigmoid and Gaussian functions are commonly used for this purpose. The representation here is generalised based on single hidden layer.

\subsubsection{Actiation functions~\textit{g}}
The activation functions are critical component of ANN models, which simulate the transformation from input layer to output layer, that is generally non-linear in nature~\cite{LeCun_2015}. Various studies~\cite{Bengio_2012,Singh_2013} have demonstrated the importance of selecting proper activation function for a given scenario, which can lead to improved feature extraction. There are a few classes of commonly used activation functions as follows:
\begin{enumerate}
\item Sigmoid function \\
Sigmoid function is often characterised by its sigmoid curve, which is bounded by a pair of horizontal asymptotes as $x$ reaches infinity. As an activate function, it often take the forms either logistic function 
\begin{equation}
    g(x) = \frac{1} {1+ e^{-x}}
\end{equation}
or shifted and scaled logistic function, which commonly referred as hyperbolic tangent
\begin{equation}
    g(x) = tanh(x) = \frac{e^x-e^{-x}} {e^x + e^{-x}}
\end{equation}
\end{enumerate}
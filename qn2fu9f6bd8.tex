\subsection{Concepts of Machine Learning}
One of the learning algorithm is artificial neural networks (ANN) as demonstrated in Figure~\ref{147379}. The very first of ANN was proposed by~\citet{McCulloch_1943}, which was inspired by the structure of neural connections. In 1960s, ANN has been developed in attempt to perci  
\par 
A typical ANN model construction possesses three structural layers, namely: one input layer, one or more hidden layers and one output layer. Each layer can have one or more nodes and each node is commonly referred as a neuron. Neurons in the each layer are either connected or partially connected with neurons at neighbour layers. The output value $O_o$ can be presented in inputs form as 

\begin{equation}
   O_o = g( \sum_{j} \sum_{i} B2*B1*W_{ij} * I_i) 
\end{equation}

where $I_i$ represents the input layer with i nodes; $O_o$ is the output layer with o nodes; $W_{i}$ is the weight of each input node $I_i$ in representing hidden node $H_j$; $W_{j}$ is the weight of each hidden node $H_j$ in representing output node $O_o$;  B1 and B2 are bias layers that apply constant values to the nodes at each subsequent layers after input layer; g is the activation function and generally is a non-linear function, sigmoid and Gaussian functions are commonly used for this purpose. The representation here is generalised based on single hidden layer.
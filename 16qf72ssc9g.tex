\subsubsection{Challenges in Model Training}

Optimization is traditionally very difficult task. Model training in machine learning avoids many difficulties by carefully construct objective function to be continuous, differentiable and convex. However, there are still a few common problems need to be discussed. 

\begin{enumerate}
    \item Ill conditioning\\
Ill conditioning is one of the most common challenges in numerical optimization. The condition number is a measurement of how sensitive the output of a model is to a small change in input. Commonly the condition number is used to measure the tolerance of a model to an input error.  In multi-dimensional space, it is common to have different curvature (defined by $2^\circ$) at different direction

Hessian matrix $\mathrm{H}$ is defined as Jacobian of the gradient, which represents curvature of the $\mathcal{Obj}(\phi)$ and takes form as 
\begin{equation}
    \mathrm{H}(f)(x)_{i,j} = \frac{\partial^2}{\partial x_i \partial x_j}f(x)
\end{equation}

    \item Saddel Points and Flat Regions\\
In high dimensional model trainings, saddle points are more commonly encounter than local extremes (\citet{DauphinPGCGB14} provided a detailed review on theory), where it is both a local minimum within one cross-section and local maximum at a different cross section (Figure~\ref{555041}). At saddle points, Hessian matrix is filled with both positive and negative eigenvalues, where saddle points sit on the direction with positive eigenvalues have local minimum while on the direction with negative eigenvalues have local maximum. \\
P

\end{enumerate}

